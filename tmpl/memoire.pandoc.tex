%% Pour voir les accents de ce fichier, assurez-vous que votre
%% éditeur de texte lise le fichier en utf-8!

%% La classe <littfra> est construite au-dessus de <amsbook>, donc
%% <amsmath>, <amsfonts> et <amsthm> sont automatiquement chargés.
%% Pour un mémoire
\documentclass[11pt,twoside,maitrise]{littfra}
%% Pour une thèse
%%\documentclass[12pt,twoside,phd]{littfra}

\usepackage[utf8]{inputenc} %Obligatoires
\usepackage[T1]{fontenc}    %

%% <lmodern> incorpore les fontes en T1, pour
%% faciliter le dépôt final. Ceci n'est pas la
%% seule option :
%%  1. Si cm-super est installé, vous pouvez enlever <lmodern>
%%     (à ce moment, la police est un peu plus fidèle
%%      au Computer Modern orginal);
%%  2. Si vous avez une police préférée, par exemple,
%%     <times> ou <euler> ou <mathpazo> (et bien d'autres),
%%     alors vous pouvez remplacer <lmodern> ci-bas.
%% Par contre, si vous faîtes face à un problème d'encapsulation
%% lors dépôt final, il se peut que la solution soit d'utiliser <lmodern>.
%% (Parfois le problème est au niveau de l'installation, donc
%%  essayez de compiler sur un autre ordinateur sur lequel vous êtes
%%  certain·e que l'installation est bonne.)
\usepackage{lmodern}

%% Il n'est pas nécessaire d'utiliser <babel>, car
%% les commandes intégrées par la classe <littfra>
%% \francais et \anglais font le travail. Néanmoins,
%% certains autres packages nécessitent <babel> (comme
%% <natbib>), donc simplement enlever les % devant <babel>
%% dans ce cas. Attention! Certains packages sont sensibles
%% à l'ordre dans lequel ils sont chargés.
\francais % or
%%\anglais
%%
%\usepackage[english,french]{babel}

% Aidons LaTeX à interpréter certains caractères unicode (surtout dans biblio/CSL)
\DeclareUnicodeCharacter{1D49}{\textsuperscript{e}} % «e» en exposant
\DeclareUnicodeCharacter{1D52}{\textsuperscript{o}} % «o» en exposant
\DeclareUnicodeCharacter{02E2}{\textsuperscript{s}} % «s» en exposant
\DeclareUnicodeCharacter{2B3}{\textsuperscript{r}} % «r» en exposant
\DeclareUnicodeCharacter{00B0}{\textsuperscript{o}} % «degré» => convertir en «o» en exposant

 % ENGLISH OPTION
 % If you call \anglais here before the \begin{document},
 % all the chater's header will be in english, even if you
 % call \francais. To change this, use
 % \entetedynamique

%% La commande \sloppy peut avoir des effets étranges sur les
%% lignes de certains paragraphes.  Dans ce cas, essayez \fussy
%% qui suppresse les effets de \sloppy.
%% (\fussy est normalement le comportement par défaut.)
%% On redéfinit \sloppy, pour tenter de réduire les comportements
%% étranges. Le seul changement apporté à la version originale
%% est la valeur de \tolerance.
\def\sloppy{%
  \tolerance 500%  %9999 dans LaTeX ordinaire, mauvaise idée.
  \emergencystretch 3em%
  \hfuzz .5pt
  \vfuzz\hfuzz}
\sloppy   %appel de \sloppy pour le document
%%\fussy  %ou \fussy

%% Packages utiles.
\usepackage{graphicx,amssymb,subfigure,icomma}

% figures
% via le modèle pandoc: https://github.com/jgm/pandoc/blob/main/data/templates/default.latex
%\usepackage{graphicx}
\makeatletter
\def\maxwidth{\ifdim\Gin@nat@width>\linewidth\linewidth\else\Gin@nat@width\fi}
\def\maxheight{\ifdim\Gin@nat@height>\textheight\textheight\else\Gin@nat@height\fi}
\makeatother
% Scale images if necessary, so that they will not overflow the page
% margins by default, and it is still possible to overwrite the defaults
% using explicit options in \includegraphics[width, height, ...]{}
\setkeys{Gin}{width=\maxwidth,height=\maxheight,keepaspectratio}
% Set default figure placement to htbp
\makeatletter
\def\fps@figure{htbp}
\makeatother
$if(svg)$
\usepackage{svg}
$endif$

% From https://github.com/jgm/pandoc/blob/main/data/templates/default.latex
$if(indent)$
$else$
\makeatletter
\@ifundefined{KOMAClassName}{% if non-KOMA class
  \IfFileExists{parskip.sty}{%
    \usepackage{parskip}
  }{% else
    \setlength{\parindent}{0pt}
    \setlength{\parskip}{6pt plus 2pt minus 1pt}}
}{% if KOMA class
  \KOMAoptions{parskip=half}}
\makeatother
$endif$
$if(verbatim-in-note)$
\usepackage{fancyvrb}
$endif$

\usepackage{xcolor}
% combiné avec xcolor, on peut utiliser la commande `hl{texte en subrillance}`
% voir https://tex.stackexchange.com/questions/114866/highlight-color-parts-of-text
\usepackage{soul}

$if(geometry)$
$if(beamer)$
\geometry{$for(geometry)$$geometry$$sep$,$endfor$}
$else$
\usepackage[$for(geometry)$$geometry$$sep$,$endfor$]{geometry}
$endif$
$endif$
$if(beamer)$
\newif\ifbibliography
$endif$
$if(listings)$
\usepackage{listings}
\newcommand{\passthrough}[1]{#1}
\lstset{defaultdialect=[5.3]Lua}
\lstset{defaultdialect=[x86masm]Assembler}
$endif$
$if(lhs)$
\lstnewenvironment{code}{\lstset{language=Haskell,basicstyle=\small\ttfamily}}{}
$endif$
$if(highlighting-macros)$
$highlighting-macros$
$endif$
% end imported chunk

% tables
$if(tables)$
\usepackage{longtable,booktabs,array}
$if(multirow)$
\usepackage{multirow}
$endif$
\usepackage{calc} % for calculating minipage widths
$endif$
%% icomma       permet d'écrire les nombres décimaux en
%%                  français (p.ex. 1,23 plutôt que 1.23)
%% subfigure    simplifie l'inclusion de figures côtes-à-côtes

%% Packages parfois utiles.
%%\usepackage{dsfont,mathrsfs,color,url,verbatim,booktabs}
%% dsfont       symboles mathématiques \mathds
%% mathrsfs     plus de symboles mathématiques \mathscr
%% color        pour utiliser des couleurs (comparer avec <xcolor>)
%% url          permet l'écriture d'url
%% verbatim     pour écrire du code ou du texte tel quel
%% booktabs     plus de macros pour faire les tableaux
%%                  (voir documentation du package)

%% pour que la largeur de la légende des figures soit = \textwidth
\usepackage[labelfont=bf, width=\linewidth]{caption}

%% les 3 lignes suivante servent à l'affichage de l'index
%% dans le visionneur de pdf. <hyperref> et <bookmark>
%% devraient être les dernier package a être chargé,
%% donc chargez vos packages avant.
\usepackage{hyperref}  % Ajoute les hyperlien
\hypersetup{colorlinks=true,allcolors=black}
\usepackage{hypcap}   % Corrige la position du lien pour les images
\usepackage{bookmark} % Remédie à des petits problème
                      % de <hyperref> (important qu'il
                      % apparaisse APRÈS <hyperref>)
\usepackage{epigraph} % Pour produire de belles citations en exergue
\setlength\epigraphrule{0pt} % Masquer la ligne verticale entre citation et auteur
\setlength\epigraphwidth{.42\textwidth} % Largeur sur mesure
\setlength\beforeepigraphskip{8ex}
\setlength\afterepigraphskip{8ex}

  % Enlever les commentaires du prochaine \hypersetup et
  % le remplir avec l'information pertinente.
  % Ceci ajoute des « méta-données » au pdf.  C'est optionnel,
  % mais recommandé. Vous pouvez voir ces méta-données en
  % ouvrant un visionneur de pdf et en cherchant les propriétés
  % du pdf. (Vous pouvez aussi tapez ' pdfinfo <nom-du-pdf> '
  % dans un terminal.) Ces données sont utiles, par exemple,
  % pour augmenter les chances qu'un algorithme de recherche
  % trouve votre document sur Internet, une fois diffusé.
\hypersetup{
  pdftitle = {$title$},
  pdfauthor = {$author$},
  pdfsubject = {$title$},
  pdfkeywords = {$for(mots-clefs)$$mots-clefs$$sep$, $endfor$}
}

%% Définition des environnements utiles pour un mémoire scientifique.
%% La numérotation est laissée à la discrétion de l'auteur·e. L'exemple
%% illustré ici produit « Définition x.y.z »
%%   x = no. chapitre
%%   y = no. section
%%   z = no. définition
%% et la numérotation des corollaires, définitions, etc. se fait
%% successivement.
%%
%% Les macros \<type>name sont telles qu'ils suivent
%% la langue actuelle. (P.ex. si \francais est utilisé,
%% alors \begin{theo} va faire un Théorème et si \anglais
%% est utilisé, \begin{theo} fera un Theorem.)
%%
%\newtheorem{cor}{\corollaryname}[section]
%\newtheorem{deff}[cor]{\definitionname}
%\newtheorem{ex}[cor]{\examplename}
%\newtheorem{lem}[cor]{\lemmaname}
%\newtheorem{prop}[cor]{Proposition}
%\newtheorem{rem}[cor]{\remarkname}
%\newtheorem{theo}[cor]{\theoremname}
%\theoremstyle{definition}
%\newtheorem{algo}[cor]{\algoname}
%% NOTE : Il peut être commode de redéfinir \the<type> pour
%% obtenir la numérotation désirée. Par exemple, pour
%% que les corollaires soit numérotés #section.#sous-section.#sous-sous-section.#paragraphe.#corollaire,
%% on fait
%% \renewcommand\thecor{\theparagraph.\arabic{cor}}

%%%
%%% Si vous préférez que les corollaires, définitions, théorèmes,
%%% etc. soient numérotés séparément, utilisez plutôt un bloc de
%%% commandes de la forme :
%%%

%%\newtheorem{cor}{\corollaryname}[section]
%%\newtheorem{deff}{\definitionname}[section]
%%\newtheorem{ex}{\examplename}[section]
%%\newtheorem{lem}{\lemmaname}[section]
%%\newtheorem{prop}{Proposition}[section]
%%\newtheorem{rem}{\remarkname}[section]
%%\newtheorem{theo}{\theoremname}[section]

%%
%% Numérotation des équations par section
%% et des  tableaux et figures par chapitre.
%% Ceci peut être modifié selon les préférences de l'utilisateur.
\numberwithin{equation}{section}
\numberwithin{table}{chapter}
\numberwithin{figure}{chapter}

%%
%% Si on veut faire un index, il faut décommenter la ligne
%% suivante. Ajouter des mots à l'index avec la commande \index{mot cle} au
%% fur et à mesure dans le texte.  Compiler, puis taper la commande
%% makeindex pour creer les indexs.  Après une nouvelle compilation,
%% vous aurez votre index.
%%

%%\makeindex

% Bibliographie
% Pour usage avec pandoc
\newlength{\cslhangindent}
\setlength{\cslhangindent}{1.5em}
\newlength{\csllabelwidth}
\setlength{\csllabelwidth}{3em}
\newlength{\cslentryspacingunit} % times entry-spacing
\setlength{\cslentryspacingunit}{\parskip}
\newenvironment{CSLReferences}[2] % #1 hanging-ident, #2 entry spacing
{% don't indent paragraphs
  \setlength{\parindent}{0pt}
  % turn on hanging indent if param 1 is 1
  \ifodd #1
  \let\oldpar\par
  \def\par{\hangindent=\cslhangindent\oldpar}
  \fi
  % set entry spacing
  \setlength{\parskip}{#2\cslentryspacingunit}
  }%
  {}
\usepackage{calc}
\newcommand{\CSLBlock}[1]{#1\hfill\break}
\newcommand{\CSLLeftMargin}[1]{\parbox[t]{\csllabelwidth}{#1}}
\newcommand{\CSLRightInline}[1]{\parbox[t]{\linewidth - \csllabelwidth}{#1}\break}
\newcommand{\CSLIndent}[1]{\hspace{\cslhangindent}#1}

$if(bibliography)$
\bibliography{$for(bibliography)$$bibliography$$sep$,$endfor$}
$endif$

%% Il est obligatoire d'écrire à double interligne
%% ou à interligne et demi. On peut soit utiliser
%% le package <setspace> ou \baselinestretch.
%% Le package est un peu plus propre, mais le choix
%% reste à la discrétion de l'usager.
\usepackage[doublespacing]{setspace}
\usepackage{biblatex}
 % ou
\renewcommand{\baselinestretch}{2}
% edit (Louis): la commande usepackage avec l’option
% doublespacing ci-haut ne suffit pas, l’interligne
% demeure à 1.5... donc il faut *aussi* utiliser
% la commande \baselinestretch{2}

%%%%%%%%%%%%%%%%%%%%%%%%%%%%%%%%%%%%%%%%%%%%%%%%%%%%%%%%%%%%
%%%%%%%%%%%%%%%%%%%%%%%%%%%%%%%%%%%%%%%%%%%%%%%%%%%%%%%%%%%%
%%%%%%%%%%                                     %%%%%%%%%%%%%
%%%%%%%%%% D é b u t    d u    d o c u m e n t %%%%%%%%%%%%%
%%%%%%%%%%                                     %%%%%%%%%%%%%
%%%%%%%%%%%%%%%%%%%%%%%%%%%%%%%%%%%%%%%%%%%%%%%%%%%%%%%%%%%%
%%%%%%%%%%%%%%%%%%%%%%%%%%%%%%%%%%%%%%%%%%%%%%%%%%%%%%%%%%%%
\begin{document}

%%
%% Voici des options pour annoter les différentes versions de votre
%% mémoire. La commande \brouillon imprime, au bas de chacune des pages, la
%% date ainsi que l'heure de la dernière compilation de votre fichier.
%%
%%\brouillon
%%
%%
%% \version est la version de votre manuscrit
%%
\version{$version$}

%%------------------------------------------------- %
%%              pages i et ii                       %
%%------------------------------------------------- %

%%%
%%% Voici les variables à définir pour les deux premières pages de votre
%%% mémoire.
%%%

\title{$title$}

$if(subtitle)$
\subtitle{$subtitle$}
$endif$

$if(author)$
\author{$author$}
$endif$

$if(copyrightyear)$
\copyrightyear{$copyrightyear$}
$endif$

$if(departement)$
\department{$departement$}
$endif$

$if(date)$
\date{$date$} %Date du DÉPÔT INITIAL (ou du 2e dépôt s'il y a corrections majeures)
$else$
% Si omission de la date, utilisons simplement celle d’aujourd’hui
\date{\todayfr} % la commande `\todayfr` est définie dans le fichier littfra.cls
$endif$

$if(discipline)$
\sujet{$discipline$}
$endif$

$if(orientation)$
%%\orientation{orientation}%Ce champ est optionnel
%%
%% Voici les disciplines possibles (voir avec votre directeur):
%% \sujet{statistique},
%% \sujet{mathématiques}, \orientation{mathématiques appliquées},
%% \orientation{mathématiques fondamentales}
%% \orientation{mathématiques de l'ingénieur} et
%% \orientation{mathématiques appliquées}
\orientation{$orientation$}
$endif$

% Nom du président du jury
$if(president)$
\president{$president$}
$endif$

% Nom du directeur de recherche
$if(directeur)$
\directeur{$directeur$}
$endif$

%%\codirecteur{Nom du 1er codirecteur}         % s'il y a lieu
%%\codirecteurs{Nom du 2e codirecteur}         % s'il y a lieu

$if(membrejury)$
\membrejury{$membrejury$}
$endif$

$if(examinateur)$
\examinateur{$examinateur$}   %Nom de l'examinateur externe, obligatoire pour la these
$endif$

$if(membresjury)$
%% % Deuxième membre du jury, s'il y a lieu
\membresjury{$membresjury$}
$endif$

$if(plusmembresjury)$
%%  % Troisième membre du jury, s'il y a lieu
\plusmembresjury{$plusmembresjury$}
$endif$

$if(repdoyen)$
 % Cette option existe encore, mais elle n'a plus sa place
 % dans la page titre. L'utiliser seulement si le directeur
 % insiste...
 %(thèse seulement)
\repdoyen{$repdoyen$}
$endif$

%%
%% Fin des variables à définir. La commande \maketitle créera votre
%% page titre.

\maketitle


% Réglages macro-typo pour le corps
% ---------------------------------
% 1. Aucune indentation
%\setlength{\parindent}{0pt}
% 2.Espacement entre les paragraphes (car pas d’indentation)
%   Set \parskip to put 12pt (+2/-1 pt) between paragraphs
%\setlength{\parskip}{13pt plus 2pt minus 1pt}
% Fin des réglages macro-typo

\setlength{\parindent}{30pt}
\setlength{\parskip}{0pt}

 % Pour générer la deuxième page titre, il faut appeler à nouveau \maketitle
 % Cette page est obligatoire.
\maketitle

\IfFileExists{./tmp/resume.md.tex}{%
  %%------------------------------------------------- %
  %%              pages iii                           %
  %%------------------------------------------------- %

  \francais
  \chapter*{Résumé}

  \begin{spacing}{1.5}
  \input{./tmp/resume.md.tex}

  $if(mots-clefs)$
  \textbf{Mots-clefs} : $for(mots-clefs)$$mots-clefs$$sep$, $endfor$.
  $endif$

  \end{spacing}
}{} % end file exists

\IfFileExists{./tmp/abstract.md.tex}{%
  %%------------------------------------------------- %
  %%              pages iv                            %
  %%------------------------------------------------- %

  \anglais
  \chapter*{Abstract}

  \begin{spacing}{1.5}
  \input{./tmp/abstract.md.tex}

  $if(keywords)$
  \textbf{Keywords}: $for(keywords)$$keywords$$sep$, $endfor$.
  $endif$
  \end{spacing}
}{} % end file exists

%%------------------------------------------------- %
%%        page v --- Table de matieres              %
%%------------------------------------------------- %

 % Pour un mémoire en anglais, changer pour
 % \anglais. Noter qu'il faut une permission
 % pour écrire son mémoire en anglais.
%%\anglais
\francais
 % \cleardoublepage termine la page actuel et force TeX
 % a poussé les éléments flottant (fig., tables, etc.) sur
 % la page (normalement TeX les garde en suspend jusqu'à ce
 % qu'il trouve un endroit approprié). Avec l'option <twoside>,
 % la commande s'assure que la prochaine page de texte est sur
 % le recto, pour l'impression. On l'utilise ici
 % pour que TeX sache que la table des matières etc. soit
 % sur la page qui suit.
%% TABLE DES MATIÈRES
\cleardoublepage
\pdfbookmark[chapter]{\contentsname}{toc}  % Crée un bouton sur
                                           % la barre de navigation
\begin{spacing}{1.15}
\tableofcontents
\end{spacing}
 % LISTE DES TABLES
\cleardoublepage
\phantomsection  % Crée une section invisible (utile pour les hyperliens)
%\listoftables
 % LISTE DES FIGURES
%\cleardoublepage
%\phantomsection
%\listoffigures


$if(abbreviations)$
%%%%%%%%%%%%%%%%%%%%%%%%%%%%%%%%%%%%%
%% LISTE DES SIGLES ET ABRÉVIATION %
%%%%%%%%%%%%%%%%%%%%%%%%%%%%%%%%%%%%%
%% Il est obligatoire, selon les directives de la FESP,
%% pour une thèse ou un mémoire d'avoir une liste des sigles et
%% des abréviations.  Si vous considérez que de telles listes ne seraient pas
%% pertinentes (si, par exemple, vous n'utilisez aucun sigle ou abré.), son
%% inclusion ou omission est laissé à votre discrétion.  En cas de doute,
%% parlez-en à votre directeur de recherche, le coadministrateur ou au/à la
%% bibliothécaire.
%%
%% Le gabarit inclut un exemple d'une liste « fait à la main ».  Il existe des outils
%% plus sophistiqués si vous devez inclure une multitude de sigles et abréviations.
%% Par exemple, le package <glossaries> peut faire des index élaborés.  Comme
%% son utilisation est technique, il n'y a pas d'exemple directement dans ce gabarit.
%% On invite les gens qui aurait à l'utiliser à lire la documentation officielle,
%% soit en allant sur https://www.ctan.org/, soit en tapant dans un terminal :
%%
%% texdoc glossaries
%%

\chapter*{Liste des sigles et des abréviations}
 % Option de colonnes: definir \colun ou \coldeux
%%% Exemple
%%% \def\colun{\bf} % Première colonne en gras
%%% Pour numéroté les entrées, on peut faire
%%% \newcount\abbrlist
%%% \abbrlist=0
%%% \def\plusun{\global\advance\abbrlist by 1\relax}
%%% \def\colun{\plusun\the\abbrlist. }
%%\def\coldeux{\relax}
\begin{twocolumnlist}{.2\textwidth}{.7\textwidth}
  $for(abbreviations)$
  $abbreviations.sigle$ & $abbreviations.complete$\\
  $endfor$
\end{twocolumnlist}
%% L'environnement <threecolumnlist> existe aussi pour trois colonnes.
$endif$

\IfFileExists{./tmp/remerciements.md.tex}{%
  %%------------------------------------------------- %
  %%              pages vi                            %
  %%------------------------------------------------- %

  \chapter*{Remerciements}

  \begin{spacing}{1.5}
  \input{./tmp/remerciements.md.tex}
  \end{spacing}
}{} % end file exists

 %
 % Fin des pages liminaires.  À partir d'ici, les
 % premières pages des chapitres ne doivent pas
 % être numérotées
 %

\NoChapterPageNumber
\cleardoublepage

%%%%%%%%%%%%%%%%%%%%%%%%%%%%%%%%%%%%%%%%%%%%%%%%%%%%%
%%                                                  %
%%   TEXTE DU MÉMOIRE :  introduction page 1,...    %
%%                                                  %
%%%%%%%%%%%%%%%%%%%%%%%%%%%%%%%%%%%%%%%%%%%%%%%%%%%%%

\IfFileExists{./tmp/introduction.md.tex}{%
  \setcounter{secnumdepth}{-1}
  \chapter*{Introduction}

  \input{./tmp/introduction.md.tex}
  \setcounter{secnumdepth}{4}
}{} % end file exists

%%------------------------------------------------- %
%%                pages 1                           %
%%------------------------------------------------- %

%\epigraph{\itshape Begin at the beginning, the King said gravely, ``and go on till you come to the end: then stop.''}{---Lewis Carroll, \textit{Alice in Wonderland}}

%\IfFileExists{./tmp/chapitres.md.tex}{%
%  \input{./tmp/chapitres.md.tex}
%}{} % end file exists

$body$

\IfFileExists{./tmp/conclusion.md.tex}{%
  \setcounter{secnumdepth}{-1}
  %% si la numérotation des notes de pas de page se fait par chapitre,
  %% on doit remet le compteur à zéro pour cette section (qui n'est pas numérotée)
  %\setcounter{footnote}{0}
  \chapter*{Conclusion}

  \input{./tmp/conclusion.md.tex}
  % On peut vouloir la numérotation à partir d’ici, ou non. Décommenter pour en avoir.
  %\setcounter{secnumdepth}{-1}
}{} % end file exists
%%--------------%
%%     index    %
%%--------------%

%% S'il y a lieu, décommenter la ligne pour mettre votre index

%%\printindex

%%------------------------------------------------- %
%%         références --- bibliographie             %
%%------------------------------------------------- %
  % Enlever les commentaires de la prochaine commande si vous préférez que le
  % chapitre s'appelle « Références » plutôt que « Bibliographie » (au choix selon le contexte).
%%\let\bibname=\refname

%% Lorsque vous serez prêt à faire afficher votre bibliographie
%% et vos références, enlevez les commandaires des commandes suivantes
%% et donnez le nom de votre fichier .bib à la commande \bibliography{..}
%% (consultez l'exemple au besoin).  Vous pouvez utiliser le style de votre
%% choix.
%\bibliographystyle{plain-fr}     % Le style de la bibliographie. Notons que
                                        % les extensions ne sont pas données pour ces deux fichiers.
%\def\bibname{R\'ef\'erences bibliographiques} % Nom obligatoire de la section des références.
                                              % On utilise \'e car le é cause des problèmes
                                              % dans la table des matière
%% ENGLISH
 %\def\bibname{References}
%\bibliography{ref}     % La base de données contenant des entrées bibliographiques.
                                    % Seules celles référencées dans le texte seront ajoutées
                                    % \`a la bibliographie.

\IfFileExists{./tmp/references.md.tex}{%
  \setcounter{secnumdepth}{-1}
  \chapter*{R\'ef\'erences bibliographiques}

  \begin{spacing}{1.35}
  \input{./tmp/references.md.tex}
  \end{spacing}
% On peut vouloir la numérotation à partir d’ici, ou non. Décommenter pour en avoir.
%\setcounter{secnumdepth}{-1}
}{} % end file exists

% Bibliographie, pour usage avec pandoc
%$if(natbib)$
%$if(bibliography)$
%$if(biblio-title)$
%$if(has-chapters)$
%\renewcommand\bibname{$biblio-title$}
%$else$
%\renewcommand\refname{$biblio-title$}
%$endif$
%$endif$
%$if(beamer)$
%\begin{frame}[allowframebreaks]{$biblio-title$}
%  \bibliographytrue
%  $endif$
%  \bibliography{$for(bibliography)$$bibliography$$sep$,$endfor$}
%  $if(beamer)$
%\end{frame}
%$endif$
%
%$endif$
%$endif$
%$if(biblatex)$
%$if(beamer)$
%\begin{frame}[allowframebreaks]{$biblio-title$}
%  \bibliographytrue
%  \printbibliography[heading=none]
%\end{frame}
%$else$
%\printbibliography$if(biblio-title)$[title=$biblio-title$]$endif$
%$endif$
%
%$endif$

%%------------------------------------------------- %
%%                  Annexe A                        %
%%------------------------------------------------- %

\appendix

\IfFileExists{./tmp/annexes.md.tex}{%
  \begin{spacing}{1.5}
  \input{./tmp/annexes.md.tex}
  \end{spacing}
}{} % end file exists


%%------------------------------------------------- %
%%                  Colophon                        %
%%------------------------------------------------- %

\newpage
\pagenumbering{gobble}
\setlength{\parindent}{0pt}

\vspace*{\fill}

\centering
\footnotesize
\textsc{cc by-sa 4.0}

La licence de libre diffusion autorise tous les usages \\
non commerciaux, avec un partage dans les mêmes conditions.

Ce mémoire est composé en Latin Modern, en recourant notamment à \\
pandoc de John McFarlane et au \LaTeX{} de Donald Knuth.


\end{document}



\endinput
%%
%% End of file `memoire.pandoc.tex'.
